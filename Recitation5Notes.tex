\documentclass[12pt]{article}
\usepackage{amsfonts, mathtools, mathrsfs, amssymb, graphicx, graphics, subfigure, lscape,enumerate, multirow, natbib}
\usepackage[margin=1in]{geometry}
\usepackage[bottom,flushmargin,hang,multiple]{footmisc}
\usepackage{etoolbox}

\makeatletter
\patchcmd\maketitle{\hb@xt@1.8em}{\hbox}{}{}
\makeatother

\makeatletter
\def\blfootnote{\xdef\@thefnmark{}\@footnotetext}
\makeatother

\newcommand{\rank}{\ensuremath{\text{rank}}}
\newcommand{\nullity}{\ensuremath{\text{nullity}}}

\newcommand{\icol}[1]{% inline column vector
	\left(\begin{smallmatrix}#1\end{smallmatrix}\right)%
}
\newcommand{\iicol}[1]{% inline column vector
	\left[\begin{smallmatrix}#1\end{smallmatrix}\right]%
}


\title{Recitation 5: More Multivariate Normal}
\date{}
\author{}

\begin{document}

\maketitle	

\vspace{-30pt}

\section{Representations, roughly}
\begin{itemize}
	\item Distributions are typically defined by functions (such as density functions, characteristic functions, moments, etc.)
	\item However, we can also represent a density in terms of other random variables\footnotemark \footnotetext{Blitzstein and Morris (upcoming book)  \hfill Joy Yang, October 12, 2018}
	\item If you're an algorithms buff, this is somewhat analogous to reducing a seemingly complicated problem to a problem in P and then realizing that there's a simple solution
\end{itemize}

\section{Representation of Multivariate Normal}
\begin{itemize}
	\item $y$ is Multivariate normal, or 
	$$y \sim N(\mu, \Sigma)$$
	if we can write 
	\begin{equation*}
	\underset{k\times 1}{y} =  \underset{k\times m}{A} \  
	\underset{m\times 1}{z} + \underset{k\times 1}{\mu}
	\end{equation*}
	$$z_i \sim N(0,1) \text{ ,\ \ and \ \ } \Sigma = AA^T$$
	Note that this means $z \sim N(0,I_{m \times m})$
	
	It's easy to show that this definition is equivalent to the function definitions of MVN.
	
	\item \textbf{Question:} Using its representation, find the mean and variance of $y$
\end{itemize}

\begin{align*}
	E(y) &= E(Az + \mu) \\
		 &= AE(z) + \mu \\
		 &= \mu
\end{align*}

\begin{align*}
 Var(y) &= Var(Az + \mu) \\
		&= A\ Var(z)\ A^T + 0 \\
		&= AA^T
\end{align*}

\newpage

\section{Linear Combinations of MVN}
\begin{itemize}
	\item \textbf{Question:} Are linear combinations of MVN random variables MVN themselves?
	\begin{align*}
	y^* &= By \\
	&= B(Az + \mu) \\
	&= BAz + B \mu
	\end{align*}
	
	Note that this is now in the form of a MVN random variable,
	\begin{align*}
	y^* &\sim N(B\mu, B A A^T B^T) \\
	y^* &\sim N(B\mu, B \Sigma B^T)
	\end{align*}	
\end{itemize}

\newpage

\section{Subvectors of MVN}
\begin{itemize}
	\item Last week, Jennifer showed that subvectors of MVN are, themselves MVN
	\begin{equation*}
	y = \left[ \begin{array}{c} y_1 \\ y_2 \end{array} \right] \sim N \left(
	\left[ \begin{array}{c} \mu_1 \\ \mu_2 \end{array} \right],
	\left[
	\begin{array}{cc}
	\Sigma_{11} & \Sigma_{12} \\
	\Sigma_{21} & \Sigma_{22}
	\end{array}
	\right]
	\right)
	\end{equation*}
	implies that $$y_1 \sim N(\mu_1, \Sigma_{11})$$
	
	\item Showing this required marginalizing out $y_2$
	$$P(y_1) = \int P(y_1,y_2) \ d y_2$$
	This got real involved real fast. We needed to work with block matrix inverses, somewhere, we did a vector complete-the-square, and then somehow, after pages of work, the integral turned out to be 1, and the constants we pulled out of the integral turned out to specify an MVN.\footnote{A nice reference: http://cs229.stanford.edu/section/more\_on\_gaussians.pdf}
	
	\item \textbf{Question:} Derive the above property using the representation of $y$.
	
	Hint: Can we write $y_1$ as a linear combination of $y$?
	$$y_1 = \left[ I \ 0\right] y $$
	So, $y_1$ is MVN
	\begin{align*}
		y_1 &\sim N (\left[ I \ 0\right] \mu, \left[ I \ 0\right] \Sigma \left[ I \ 0\right]^T) \\
		y_1 &\sim N(\mu_1, \Sigma_{11})
	\end{align*}
	
	
\end{itemize}




\newpage

\section{The Uncorrelation Trick}

\begin{itemize}
	\item Last week, Jennifer also showed that conditional distributions of MVN are MVN
	\begin{equation*}
	y = \left[ \begin{array}{c} y_1 \\ y_2 \end{array} \right] \sim N \left(
	\left[ \begin{array}{c} \mu_1 \\ \mu_2 \end{array} \right],
	\left[
	\begin{array}{cc}
	\Sigma_{11} & \Sigma_{12} \\
	\Sigma_{21} & \Sigma_{22}
	\end{array}
	\right]
	\right)
	\end{equation*}
	implies that $$y_2 | y_1 \sim N(\mu^*, \Sigma^*)$$
	\begin{align*}
	\text{where \ \ } \mu^* &= \mu_2 + \Sigma_{21} \Sigma_{11}^{-1} (y_1 - \mu_1) \\
	 \Sigma^* &= \Sigma_{22} - \Sigma_{21} \Sigma_{11}^{-1} \Sigma_{12}
	\end{align*}	
	\item Showing this required using the law of conditional probabilities
	$$P(y_2 | y_1) = \frac{P(y_1, y_2)}{P(y_1)}$$
	Again, this got real involved real fast. We again needed to use block matrix inverses, complete-the-square, etc.
	\item Instead, today, we'll show this using the ``uncorrelation trick"\footnote{Blitzstein \& Morris, upcoming book; other handy references: \textit{Multivariate Analysis} (Mardia, Kent, Bibby), or Stanford Stat 306 Notes - http://statweb.stanford.edu/~lpekelis/306a/}:
	
	We can write $y_2$ in terms of $y_1$ and $y_{2 \cdot 1}$, a centered MVN that's independent of $y_1$
	\begin{equation*}
	y_2 =  \underset{\text{independent of }y_1}{y_{2 \cdot 1} + \mu_2} + \underset{\text{function of }y_1}{B(y_1 - \mu_1)}
	\end{equation*}
	But what is $B$?
	\begin{align*}
		cov(y_2,y_1) &= cov(y_{2 \cdot 1} + \mu_2 + B(y_1 - \mu_1), y_1) \\
		cov(y_2,y_1) &= B cov(y_1, y_1) \\
		\Sigma_{21} &= B \Sigma_{11} \\
		B &= \Sigma_{21} \Sigma_{11}^{-1} 
	\end{align*}
	In lecture, Stefanie mentioned a few times that getting this term is like doing linear regression. We're essentially regressing $y_2$ against $y_1$.
	
	Taking $\tilde{y}_1 = y_1 - \mu_1$ and $\tilde{y}_2 = y_2 - \mu_2$, we can write down a model
	\begin{align*}
	\tilde{y}_2 &= B\tilde{y}_1 + \epsilon
	\end{align*}
	This leads us to the interpretation that $y_{2 \cdot 1}$ is then the residual $\epsilon$ after linear regression
	
	And the least squares solution for B (see section 2 notes) is
	\begin{align*}
	B &= \tilde{y}_2 \tilde{y}_1^T (\tilde{y}_1 \tilde{y}_1^T)^{-1} \\
	  &= \Sigma_{21} \Sigma_{11}^{-1}
	\end{align*}
\end{itemize}

\section{Conditional Distributions}
\begin{itemize}
	\item Reminding ourselves of the uncorrelation trick:
	$$y_2 =  \underset{\text{independent of }y_1}{y_{2 \cdot 1} + \mu_2} + \underset{\text{function of }y_1}{B(y_1 - \mu_1)}$$
	In light of this decomposition, $y_2 | y_1$ must be MVN.
	\item It's straightforward now to find the conditional expectation and variance
	\begin{align*}
		E(y_2 | y_1) &= E(y_{2 \cdot 1} + \mu_2 + B(y_1 - \mu_1) \ | \ y_1) \\
					&= E(y_{2 \cdot 1} | y_1) + \mu_2 + B(E(y_1 | y_1) - \mu_1) \\
					&= \mu_2 + \Sigma_{21} \Sigma_{11}^{-1} (y_1 - \mu_1)
	\end{align*}
	\begin{align*}
	Var(y_2 | y_1) &= Var(y_{2 \cdot 1} + \mu_2 + B(y_1 - \mu_1) \ |\  y_1) \\
	&= Var(y_{2 \cdot 1} | y_1) + Var(\mu_2 + B(y_1 - \mu_1) | y_1) \\
	&= Var(y_{2 \cdot 1})
	\end{align*}
	So now, in order to find $Var(y_2 | y_1)$, we just need to find $Var(y_{2\cdot1})$
	\begin{align*}
		Var(y_2) &= Var(y_{2 \cdot 1} + \mu_2 + B(y_1 - \mu_1)) \\
		Var(y_2) &= Var(y_{2 \cdot 1}) + BVar(y_11)B^T \\
		\Sigma_{22} &= Var(y_{2 \cdot 1}) + \Sigma_{21} \Sigma_{11}^{-1} \Sigma_{11} \Sigma_{11}^{-1} \Sigma_{12} \\
		Var(y_{2 \cdot 1}) &= \Sigma_{22} - \Sigma_{21} \Sigma_{11}^{-1} \Sigma_{12}
	\end{align*}
	And so plugging this back in
	$$Var(y_2 | y_1) = \Sigma_{22} - \Sigma_{21} \Sigma_{11}^{-1} \Sigma_{12}$$
\end{itemize}

\newpage

\section{Sherman-Morrison-Woodbury Formula}
\begin{itemize}
	\item This might come in handy for your problem sets depending on what route you choose to go for the first question
	\item The formula:
	$$ (Z + UWV^T)^{-1} = Z^{-1} - Z^{-1}U(W^{-1} + V^T Z^{-1} U)^{-1} V^T Z^-1 $$
	\item Applications: Recursive least squares, low rank decompositions, etc. Here's a very short application, just to show how it can be useful.
	\item \textbf{Example:} Matrix perturbations
		
	Suppose the inverse of a matrix $A$ is known.
	
	We perturb each element to get a new matrix $B$
	\begin{align*}
		B_{ij} &= A_{ij} + \Delta x_i \Delta y_i \\
		B &= A + xy^T
	\end{align*}
	As a simple, concrete example, perhaps
	\begin{align*}
	x = \left[ \begin{array}{c}
	0 \\ 0 \\ 1 \\ 0
	\end{array}\right]
	\text{\ \ and \ \ }
	y = \left[ \begin{array}{c}
	1 \\ 0 \\ 0 \\ 0
	\end{array}\right]
	\text{\ \ then the perturbation matrix is \ \ }
	y = \left[ \begin{array}{cccc}
	0 & 0 & 0 & 0 \\ 0 & 0 & 0 & 0 \\ 1 & 0 & 0 & 0 \\ 0 & 0 & 0 & 0
	\end{array}\right]
	\end{align*}
	If we want to invert B, we don't need to re-compute the entire matrix inverse. Instead, just plug into the Woodbury formula:
	$$B^{-1} = A^{-1} - A^{-1}x(1 + y^T A^{-1} x)^{-1} y^T A^{-1}$$
	Notice, $(1 + y^T A^{-1} x)$ is a scalar (in this case, it's simply $1$), so that's easy to invert.

\end{itemize}

\end{document}